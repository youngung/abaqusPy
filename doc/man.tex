\documentclass[12pt]{amsart}
%\usepackage[intlimits]{amsmath}
\usepackage{setspace}
\usepackage{graphicx}
\usepackage{hyperref}
\usepackage{cite}
%\usepackage[]{natbib}
\usepackage{subfigure}
\usepackage{listings}
\usepackage{color}
%\usepackage[nomarkers,figuresonly]{endfloat}
\usepackage{geometry} % see geometry.pdf on how to lay out the page. There's lots.
\usepackage{paralist}
\usepackage{lscape}
\usepackage{caption}
\usepackage{lineno}
%\linenumbers

\usepackage{pgf}
\usepackage{tikz}
\usepackage{pgfplots}
%\usepackage{harvard}
\usetikzlibrary{shapes,arrows,chains,automata,fit}
\usetikzlibrary{positioning}
\usetikzlibrary{shapes.geometric,intersections}

\author{Youngung Jeong}
% \address{
%   International Center for Automotive Research
%   Clemson University
% }
\email[Y. Jeong]{youngung.jeong@gmail.com}

%\usepackage{authblk}
\geometry{a4paper} % or letter or a5paper or ... etc
% \geometry{landscape} % rotated page geometry
\hypersetup{
  colorlinks,%
  citecolor=blue,%,black
  filecolor=blue,%
  linkcolor=blue,%
  urlcolor=blue
}

\doublespacing


\title{Manual for abaqusPy}
%\date{} % delete this line to display the current date

%%% BEGIN DOCUMENT

\begin{document}
\pgfplotsset{compat=1.14}
\section{Microstructure deviator}
\label{sec:micd}
In the Homogeneous Anisotropic Hardening model proposed by Barlat et al. \cite{barlat2011alternative}, the dynamic changes of anisotropic hardening is described by introducing distortional hardening.
The HAH model mimics the progressive evolution of microstructures by allowing distortions of yield surface shape along particular direction that is guided by empirical parameter called microstructure deviator (denoted as $\mathbf{h}$).
Under monotnic loading, the microstructure deviator stays invariable thus keep pointing to a fixed direction in the stress space.
Note that the 'direction' of the microstructure deviator is often used, which is denoted with the hat operatore, e.g., $\hat{\mathbf{h}}$.
On the other hand, when loading direction is not monotonic but undergoes, say, an abrupt change, the microstructure deviator may also evolve into a new direction.
The two remaining questions are then 1) to which new direction should the microstructure deviator rotates and 2) in what way should this 'rotation' occur.
These questions are very difficult to answer as the microstructure deviator does not correspond to a specific measurable material property because the microstructure deviator is rather a conceptual entity introduced to {\emph effectively} capture the progressive nature of the evolutionary behavior of microstuctrure.
\newline
Nevertheless, a proper destination should be specified when the change in direction is required.
For the sake of argument, we call this new direction of the microstructure deviator the {\emph target} direction (i.e., $\hat{\mathbf{t}}$).
This direction was originally proposed to align with the direction of applied stress tensor in the stress space \cite{barlat2011alternative}.
More recently, \cite{Jeong2016} reported that the direction of the applied strain tensor might be more suitable considering the effect of anisotropy pertaining to the crystallographic texture.
To tackle the second question, Barlat et al. \cite{barlat2011alternative} proposed to use the double product $\hat{\mathbf{t}}:\hat{\mathbf{h}}$ as the measure of speed of rotation.
More specifically, this has been characterized by the $\cos \chi$ value, which resembles the parameter originally introduced by Schmitt et al. \cite{schmitt1994parameter}:
\begin{equation}
  \label{eq:coschi}
\cos\chi = \frac{\hat{\mathbf{t}}:\hat{\mathbf{h}} }{ \sqrt{\hat{\mathbf{t}}:\hat{\mathbf{t}}}\sqrt{\hat{\mathbf{h}}:\hat{\mathbf{h}}}}.
\end{equation}
Furthermore, in the most recent version of the HAH model \cite{barlat2014}, the evolutionary behavior of the microstructure deviator is written with respect to the equivalent strain increment such that
\begin{equation}
  \label{eq:dmicro1}
  \frac{\partial\hat{\mathbf{h}}}{\partial \bar{\varepsilon}} = k_{R1}\ \text{sgn}(\cos\chi)\big[\big|\frac{\cos\chi}{H}\big|^{1/k_{R2}} +g_R \big] \big(\hat{\mathbf{t}}-\cos\chi \hat{\mathbf{h}}\big),
\end{equation}
where $g_R$ evolves according to
\begin{equation}
  \label{eq:dmicro2}
\frac{dg_R}{d\bar{\varepsilon}} = k_{R3} \big[k_{R4} (1-\cos^2\chi) -g_R \big].
\end{equation}
\newline
\section{Linear transformations of stress tensor}
\label{sec:stress_deriv}
To introduce the {\emph distortions} to the yield surface, a few types of transformation of the applied stress tensor are considered.
As mentioned earlier, such distortion is guided by the microstructure deviator.
In what follows the definitions of the stress transformation and their derivatives with respect to the applied stress tensor are derived.
Note that the Voigt notation is used where plausible.
\newline
In the original version of the HAH model, the stress tensor is decomposed into two contributions: namely, co-linear and orthogonal components with respect to the microstructure deviator.
The col-linear component is defined as below:
\begin{equation}
  \label{eq:decomp1}
  \mathbf{s}_c = \frac{8}{3} (\mathbf{s}\cdot\hat{\mathbf{h}}) \mathbf{h}
\end{equation}
The derivative of $\mathbf{s}_c$ with respect to $\mathbf{s}$ is
\begin{eqnarray}
  \begin{split}
  \label{eq:decomp2}
  \frac{\partial\mathbf{s}_c}{\partial\mathbf{s}} &= \frac{8}{3}\bigg\{  \frac{\partial(\mathbf{s}\cdot\hat{\mathbf{h}})}{\partial\mathbf{s}} \otimes \hat{\mathbf{h}}+ (\mathbf{s}\cdot\hat{\mathbf{h}}) \frac{\partial\hat{\mathbf{h}}}{\partial\mathbf{s}}\bigg\}\\
  \bigg( \frac{\partial s_c}{\partial s}\bigg)_{ij}&= \frac{8}{3}\Bigg\{ \bigg(\frac{\partial(s_{k}\hat{h}_{k})}{\partial s_i}\bigg)\hat{h}_j +(s_k\hat{h}_k)\frac{\partial\hat{h}_i}{\partial s_j} \Bigg\}
  \end{split}
\end{eqnarray}
In the plane stress space only three stress components (i.e., $\sigma_{11}, \sigma_{22},$ and $\sigma_{12}$) are relevant, which then leads to four non-zero components in the deviatoric space, i.e., $s_{11}, s_{22}, s_{33}$ (i.e., $s_{33}=-s_{11}-s_{22}$) $s_{12}$.
Therefore, $\dfrac{\partial(s_k\hat{h}_k)}{\partial s_i}$ expands to
\begin{equation}
  \label{eq:dsh_ds}
  \dfrac{\partial(s_1\hat{h}_1+s_2\hat{h}_2+s_3\hat{h}_3+2s_6\hat{h}_6)}{\partial s_i}=
  \begin{bmatrix}
    \hat{h}_1&\hat{h}_2&\hat{h}_3&2\hat{h}_6
  \end{bmatrix}
\end{equation}
To that end, Eq. \ref{eq:decomp2} can be expressed as:
\begin{equation}
  \label{eq:dsc_ds_matrix}
  \begin{split}
    &\begin{bmatrix}
      \bigg( \dfrac{\partial s_c}{\partial s}\bigg)_{11}&\bigg( \dfrac{\partial s_c}{\partial s}\bigg)_{12} &\bigg( \dfrac{\partial s_c}{\partial s}\bigg)_{13}&\bigg( \dfrac{\partial s_c}{\partial s}\bigg)_{16}\\
      \bigg( \dfrac{\partial s_c}{\partial s}\bigg)_{21}&\bigg( \dfrac{\partial s_c}{\partial s}\bigg)_{22} &\bigg( \dfrac{\partial s_c}{\partial s}\bigg)_{23} &\bigg( \dfrac{\partial s_c}{\partial s}\bigg)_{26}\\
      \bigg( \dfrac{\partial s_c}{\partial s}\bigg)_{31}&\bigg( \dfrac{\partial s_c}{\partial s}\bigg)_{32} &\bigg( \dfrac{\partial s_c}{\partial s}\bigg)_{33}&\bigg( \dfrac{\partial s_c}{\partial s}\bigg)_{36}\\
      \bigg( \dfrac{\partial s_c}{\partial s}\bigg)_{61}&\bigg( \dfrac{\partial s_c}{\partial s}\bigg)_{62} &\bigg( \dfrac{\partial s_c}{\partial s}\bigg)_{63}&\bigg( \dfrac{\partial s_c}{\partial s}\bigg)_{66}
    \end{bmatrix}\\
    &=\dfrac{8}{3}
    \begin{bmatrix}
      \hat{h}_1\hat{h}_1&\hat{h}_1\hat{h}_2&\hat{h}_1\hat{h}_3&2\hat{h}_1\hat{h}_6\\
      \hat{h}_2\hat{h}_1&\hat{h}_2\hat{h}_2&\hat{h}_2\hat{h}_3&2\hat{h}_2\hat{h}_6\\
      \hat{h}_3\hat{h}_1&\hat{h}_3\hat{h}_2&\hat{h}_3\hat{h}_3&2\hat{h}_3\hat{h}_6\\
      \hat{h}_6\hat{h}_1&\hat{h}_6\hat{h}_2&\hat{h}_6\hat{h}_3&2\hat{h}_6\hat{h}_6
    \end{bmatrix}
    +(s_k\hat{h}_k)\dfrac{\partial\hat{h}_i}{\partial s_j}
  \end{split}
\end{equation}
The orthogonal component is the remaining portion as defined below:
\begin{eqnarray}
  \label{eq:decomp3}
  \mathbf{s}_o = \mathbf{s} - \mathbf{s}_c.
\end{eqnarray}
The derivative of $\mathbf{s}_o$ with respect to $\mathbf{s}$ is then
\begin{eqnarray}
  \label{eq:decomp4}
  \begin{split}
    \bigg(\frac{\partial s_o}{\partial s}\bigg)_{ij} &= \delta_{ij}- \bigg(\frac{\partial s_c}{\partial s}\bigg)_{ij}.
  \end{split}
\end{eqnarray}
The latent hardening is accounted for by using a separate linear-transformation of the deviatoric stress tensor:
\begin{eqnarray}
  \label{eq:latent_decompose1}
  \mathbf{s}^{\prime\prime}=\mathbf{s}_c+\frac{1}{g_L}\mathbf{s}_o
\end{eqnarray}
Its derivative is obtained as
\begin{eqnarray}
  \begin{split}
    \label{eq:latent_decompose2}
    \frac{\partial\mathbf{s}^{\prime\prime}}{\partial\mathbf{s}} &=\frac{\partial\mathbf{s}_c}{\partial\mathbf{s}}-g_L^{-2}\frac{\partial g_L}{\partial \mathbf{s}} +g_L^{-1} \frac{\partial\mathbf{s}_o}{\partial\mathbf{s}}.
  \end{split}
\end{eqnarray}
The stress used for cross hardening goes through below transformation.
\begin{eqnarray}
  \label{eq:cross_linear1}
  \mathbf{s}_p = 4(1-g_S)\mathbf{s}_o
\end{eqnarray}
Its derivative with respect to the deviatoric stress can be obtained as:
\begin{eqnarray}
  \begin{split}
    \label{eq:cross_linear2}
    \frac{\partial\mathbf{s}_p}{\partial\mathbf{s}\hfill}  &= -4\mathbf{s}_o\otimes\frac{\partial g_S}{\partial \mathbf{s}}+4(1-g_S) \frac{\partial\mathbf{s}_o}{\partial\mathbf{s}} \\
    \bigg(\frac{\partial s_p}{\partial s\hfill} \bigg)_{ij} &= -4(s_o)_i\frac{\partial g_S}{\partial s_j} + 4(1-g_S) \bigg(\frac{\partial s_o}{\partial s}\bigg)_{ij}\\
    &= -4(s_o)_i\otimes\frac{\partial g_S}{\partial \bar{\varepsilon}}\bigg(\frac{\partial\bar{\varepsilon}}{\partial\sigma}\bigg)_k\bigg(\frac{\partial\sigma}{\partial s}\bigg)_{kj} + 4(1-g_S) \bigg(\frac{\partial s_o}{\partial s}\bigg)_{ij}
  \end{split}
\end{eqnarray}
Now, the derivatives with respect to Cauchy stress is given below,
\begin{equation}
  \label{eq:dsp_dsig}
  \begin{split}
    \bigg(\frac{\partial s_c}{\partial \sigma}\bigg)_{ij}&=\frac{8}{3}\Bigg\{\Bigg( \frac{\partial (s_m\hat{h}_m)}{\partial s_i}\hat{h}_k\frac{\partial s_k}{\partial\sigma_j} +(s_m\hat{h}_m) \bigg(\frac{\partial\hat{h}}{\partial\bar{\varepsilon}}\bigg)_i\bigg(\frac{\partial\bar{\varepsilon}}{\partial\sigma}\bigg)_j\Bigg)\Bigg\}\\
    \bigg(\frac{\partial s_o}{\partial \sigma}\bigg)_{ij}&=\bigg(\frac{\partial s}{\partial\sigma}\bigg)_{ij}-\bigg(\frac{\partial s_c}{\partial \sigma}\bigg)_{ij}\\
    \bigg(\frac{\partial s_p}{\partial \sigma}\bigg)_{ij}&=\bigg(\frac{\partial s_p}{\partial s}\bigg)_{ik} \bigg(\frac{\partial s}{\partial \sigma}\bigg)_{kj}= -4(s_o)_i\otimes\frac{\partial g_S}{\partial \bar{\varepsilon}}\bigg(\frac{\partial\bar{\varepsilon}}{\partial\sigma}\bigg)_j + 4(1-g_S) \bigg(\frac{\partial s_o}{\partial \sigma}\bigg)_{ij}\\
    \bigg(\frac{\partial s^{\prime\prime}}{\partial\sigma}\bigg)_{ij}&=\bigg(\frac{\partial s_c}{\partial\sigma}\bigg)_{ij}-g_L^{-2}\frac{\partial g_L}{\partial\bar{\varepsilon}}\bigg(\frac{\partial\bar{\varepsilon}}{\partial\sigma}\bigg)_{i}(\mathbf{s}_o)_j+g_L^{-1}\bigg(\frac{\partial s_o}{\partial\sigma}\bigg)_{ij}
  \end{split}
\end{equation}
where $\dfrac{\partial g_L}{\partial\bar{\varepsilon}}$ is given as:
\begin{equation}
  \label{eq:dgl_de}
  \dfrac{\partial g_L}{\partial\bar{\varepsilon}}=k_L\bigg[\frac{\bar{\sigma}(\bar{\varepsilon})-\bar{\sigma}(0)}{\bar{\sigma}(\bar{\varepsilon})}
  \Big(\sqrt{L(1-\cos^2\chi)+\cos^2\chi)}-1\Big)+1-g_L\bigg].
\end{equation}
Note that in the recent work of Jeong et al. \cite{Jeong2016}, it was found that the target direction $\hat{\mathbf{t}}$ should be parallel to the direction of $\dot{\mathbf{\varepsilon}}$ to properly account for the effects of crystallographic texture on Bauschinger effect.
In the associated flow rule, $\dot{\mathbf{\varepsilon}}=\dot{\lambda}\dfrac{\partial\Phi}{\partial\mathbf{\sigma}}$ thus $\hat{\mathbf{t}}=\hat{\dfrac{\partial\Phi}{\partial\mathbf{\sigma}}}$.
As a result, Eq. \ref{eq:dmicro1} is expressed as:
\begin{equation}
  \label{eq:dmicro3}
  \frac{\partial\hat{\mathbf{h}}}{\partial \bar{\varepsilon}} = k_{R1}\ \text{sgn}(\cos\chi)\bigg[\Big|\frac{\cos\chi}{H}\Big|^{1/k_{R2}} +g_R \bigg] \bigg(\hat{\dfrac{\partial\Phi}{\partial\mathbf{\sigma}}}-\cos\chi \hat{\mathbf{h}}\bigg),
\end{equation}

\newpage
\section{Derivatives associated with Bauschinger effect}
\label{sec:bauschinger}
The term $\partial f_k/\partial\mathbf{\sigma}$ depends on the evolutionary behavior of $f_k$ with respect to equivalent strain $\bar{\varepsilon}$.
This allows the application of the chain rule such as
\begin{eqnarray}
  \label{eq:dphib7}
  \frac{\partial f_k}{\partial\mathbf{\sigma}}=\frac{\partial f_k}{\partial g_k} \frac{\partial g_k}{\partial \bar{\varepsilon}}  \frac{\partial{\bar{\varepsilon}}}{\partial\mathbf{\sigma}}.
\end{eqnarray}
\begin{eqnarray}
  \label{eq:fk1}
  f_k = \bigg[ \frac{\sqrt{6H}}{4}   \Big(  g_k^{-q} - 1  \Big)  \bigg] ^{1/q}
\end{eqnarray}
\begin{eqnarray}
  \label{eq:fk2}
  \begin{split}
  \frac{\partial f_k}{\partial{g_k}} &= \frac{1}{q}   \bigg(\frac{\sqrt{6H}}{4} \bigg)^{1/q}  (-q) g_k^{-q-1}    f_k^{1-q}\\
  &=-g_k^{-q-1} f_k^{1-q}  \ \text{  if } H=8/3
  \end{split}
\end{eqnarray}
The term $  \partial{g_k}/\partial{\bar{\varepsilon}} $ depends on the sign of $\hat{\mathbf{h}}:\mathbf{s}$.
For the case of $g_1$ and when  $\hat{\mathbf{h}}:\mathbf{s}<0$:
\begin{eqnarray}
  \label{eq:dbau2}
\frac{\partial{g_1}}{\partial{\bar{\varepsilon}}}=k_1 \frac{g_4-g_1}{g_1}
\end{eqnarray}
whereas, if $\hat{\mathbf{h}}:\mathbf{s}\ge0$,
\begin{eqnarray}
  \label{eq:dbau3}
\frac{\partial{g_1}}{\partial{\bar{\varepsilon}}}=k_2 (k_3 H(0)/H(\bar{\varepsilon})-g_1)
\end{eqnarray}
On the other hand, $g_2$ follows a similar rule as follows:
when  $\hat{\mathbf{h}}:\mathbf{s}<0$:
\begin{eqnarray}
  \label{eq:dbau4}
  \frac{\partial{g_2}}{\partial{\bar{\varepsilon}}}=k_2 (k_3 H(0)/H(\bar{\varepsilon})-g_2)
\end{eqnarray}
whereas, if $\hat{\mathbf{h}}:\mathbf{s}\ge0$,
\begin{eqnarray}
  \label{eq:dbau5}
  \frac{\partial{g_2}}{\partial{\bar{\varepsilon}}}=k_1 \frac{g_3-g_2}{g_2}
\end{eqnarray}
With explicit indices, the above is expressed as:
\begin{equation}
  \label{eq:dphib8}
  \bigg(\frac{\partial f_k}{\partial \sigma}\bigg)_i = \frac{\partial f_k}{\partial g_k} \frac{\partial g_k}{\partial \bar{\varepsilon}}  \bigg(\frac{\partial\bar{\varepsilon}}{\partial\sigma}\bigg)_i.
\end{equation}
\newline
\section{Yield surface derivative}
The yield surface of the HAH model is decomposed into two contributions: the homogeneous and fluctuating terms that are denoted as $\phi_h$ and $\phi_b$, respectively.
\begin{eqnarray}
  \label{eq:hah_def}
  \Phi(\mathbf{\sigma})=\{\phi_h^q + \phi_b^q \}^{1/q}=\bar{\sigma}
\end{eqnarray}
The homogeneous term is further decomposed into two contributions:
\begin{eqnarray}
  \label{eq:hah_homo}
  \phi_h = \big(\psi(\mathbf{s}^{\prime\prime})^2+\psi(\mathbf{s}_p)^2\big)^{1/2}
\end{eqnarray}
Whereas, the fluctuating term is defined as:
\begin{eqnarray}
    \label{eq:fluc}
\phi_b  = f_1|\hat{\mathbf{h}}:\mathbf{s} - |\hat{\mathbf{h}}:\mathbf{s}||    + f_2|\hat{\mathbf{h}}:\mathbf{s} + |\hat{\mathbf{h}}:\mathbf{s}||.
\end{eqnarray}
The derivative of the HAH yield surface is:
\begin{eqnarray}
  \begin{split}
  \label{eq:hah_deriv}
  \frac{\partial\Phi}{\partial\mathbf{\sigma}} &=\frac{1}{q}\Phi^{(1-q)} \bigg\{                  \frac{\partial{\phi_h^q}}{\partial{\sigma}}  +                 \frac{\partial{\phi_b^q}}{\partial{\sigma}}         \bigg\}\\
  \bigg(\frac{\partial\Phi}{\partial\sigma}\bigg)_i&=\frac{1}{q}\Phi^{(1-q)} \bigg\{  q \phi_h^{(q-1)}\Big(\frac{\partial{\phi_h}  }{\partial{\sigma}}\Big)_i  + q \phi_b^{(q-1)} \Big(\frac{\partial{\phi_b}  }{ \partial\sigma}\Big)_i         \bigg\}\\
                                                      &=\Phi^{(1-q)} \bigg\{    \phi_h^{(q-1)}\Big(\frac{\partial{\phi_h}  }{\partial{\sigma}}\Big)_i  + \phi_b^{(q-1)} \Big(\frac{\partial{\phi_b}  }{ \partial\sigma}\Big)_i         \bigg\}
  \end{split}
\end{eqnarray}
The terms $\dfrac{\partial\phi_h}{\partial\mathbf{\sigma}}$ and $\dfrac{\partial\phi_b}{\partial\mathbf{\sigma}}$ are obtained in the following sections.
\subsection{Derivatives of homogeneous term}
\label{sec:2.1}
The derivative of the homogeneous term is obtained using the chain rule as follows:
\begin{eqnarray}
  \label{eq:derv1}
  \frac{\partial\phi_h}{\partial\mathbf{\sigma}}  =   \frac{\partial\phi_h}{\partial\mathbf{s}} \frac{\partial\mathbf{s}}{\partial\mathbf{\sigma}}.
\end{eqnarray}
The term $\partial\phi_h/\partial \mathbf{s} $ is obtained using Eq. \ref{eq:hah_homo}:
\begin{eqnarray}
  \label{eq:derv2}
  \frac{\partial{\phi_h}  }{\partial{\mathbf{s}}} = \frac{1}{2\phi_h}\bigg\{ 2\psi(\mathbf{s}^{\prime\prime})\frac{\partial{\psi(\mathbf{s}^{\prime\prime})}}{\partial{\mathbf{s}^{\prime\prime}}}  \frac{\partial\mathbf{s}^{\prime\prime}}{\partial \mathbf{s}}  +  2\psi(\mathbf{s}_p) \frac{\partial{\psi(\mathbf{s}_p)}}{\partial{\mathbf{s}_p}} \frac{\partial{\mathbf{s}_p}}{\partial{\mathbf{s}}}\bigg\}
\end{eqnarray}
By combining Eqs. \ref{eq:derv1} and \ref{eq:derv2}, below equation is obtained:
\begin{eqnarray}
  \label{eq:derv3}
  \begin{split}
  \frac{\partial{\phi_h}}{\partial{\mathbf{\sigma}}} &=\frac{1}{2\phi_h}\bigg\{ 2\psi(\mathbf{s}^{\prime\prime})\frac{\partial{\psi(\mathbf{s}^{\prime\prime})}}{\partial{\mathbf{s}^{\prime\prime}}}  \frac{\partial\mathbf{s}^{\prime\prime}}{\partial \mathbf{s}} + 2\psi(\mathbf{s}_p) \frac{\partial{\psi(\mathbf{s}_p)}}{\partial{\mathbf{s}_p}} \frac{\partial{\mathbf{s}_p}}{\partial{\mathbf{s}}}\bigg\} \frac{\partial{s}}{\partial{\sigma}}.\\
  \Big(\frac{\partial{\phi_h}}{\partial\sigma}\Big)_i &= \frac{1}{\phi_h}\bigg\{ \psi(s^{\prime\prime})\Big(\frac{\partial{\psi(s^{\prime\prime})}}{\partial{s^{\prime\prime}}}\Big)_{j}  \Big(\frac{\partial s^{\prime\prime}}{\partial\sigma}\Big)_{ji}+\psi(s_p) \Big(\frac{\partial{\psi(s_p)}}{\partial{s_p}}\Big)_{j} \Big(\frac{\partial{s_p}}{\partial\sigma}\Big)_{ji}\bigg\}
  \end{split}
\end{eqnarray}
Find the definitions of $\dfrac{\partial\mathbf{s}^{\prime\prime}}{\partial\mathbf{\sigma}}$ and $\dfrac{\partial\mathbf{s}_p}{\partial\mathbf{\sigma}}$ in Eq. \ref{eq:dsp_dsig}, respectively.
\subsection{Derivatives of fluctuating term $\partial\phi_b/\partial\mathbf{\sigma}$}
\label{sec:2.2}
The term $\phi_b$ depends on the sign of $\hat{\mathbf{h}}\cdot\mathbf{s}$ as shown in Eq. \ref{eq:fluc}.
When $\hat{\mathbf{h}}\cdot\mathbf{s}\ge0$, Eq. \ref{eq:fluc} reduces to
\begin{eqnarray}
    \label{eq:fluc1}
\phi_b =f_2|2\hat{\mathbf{h}}\cdot\mathbf{s}|=f_2|2\hat{h}_{i}\cdot s_{i}|
\end{eqnarray}
When $\hat{\mathbf{h}}:\mathbf{s}<0$, Eq. \ref{eq:fluc} reduces to
\begin{eqnarray}
    \label{eq:fluc2}
\phi_b  =f_1|2\hat{\mathbf{h}}\cdot\mathbf{s}|=f_1|2\hat{h}_{i}\cdot s_{i}|.
\end{eqnarray}
Generally,
\begin{eqnarray}
    \label{eq:fluc3}
\phi_b  =f_k|2\hat{\mathbf{h}}\cdot\mathbf{s}|=2f_k|\hat{h}_1s_1+\hat{h}_2s_2+\hat{h}_3s_3+2\hat{h}_6s_6|,
\end{eqnarray}
with the index $k$ being dependent on the sign of $\hat{\mathbf{h}}\cdot\mathbf{s}$.
The derivative $\partial\phi_b/\partial\mathbf{s}$ can be expressed as:
\begin{eqnarray}
    \label{eq:dphib1}
    \frac{\partial\phi_b}{\partial\mathbf{\sigma}} = 2\frac{\partial f_k}{\partial\mathbf{\sigma}}|\hat{\mathbf{h}}\cdot\mathbf{s}| + 2 f_k \frac{\partial|\hat{\mathbf{h}}\cdot\mathbf{s}|}{\partial\mathbf{\sigma}}
\end{eqnarray}
The term $\partial|\hat{\mathbf{h}}\cdot\mathbf{s}|/\partial\mathbf{\sigma}$ takes on either the below depending on the sign of $\hat{\mathbf{h}}\cdot\mathbf{s}$.
\begin{eqnarray}
  \label{eq:dphib2}
  \frac{\partial |\hat{\mathbf{h}}\cdot\mathbf{s}| }{\partial \mathbf{\sigma}} &=&  \frac{\partial \big(\hat{\mathbf{h}}\cdot\mathbf{s}\big)}{\partial\mathbf{\sigma}}   \text{  when } \hat{\mathbf{h}}\cdot\mathbf{s}\ge0\\
  \label{eq:dphib2_}
&=&- \frac{\partial \big(\hat{\mathbf{h}}\cdot\mathbf{s}\big)}{\partial\mathbf{\sigma}}   \text{  when } \hat{\mathbf{h}}\cdot\mathbf{s} < 0
\end{eqnarray}
Substituting Eqs. \ref{eq:dphib2} and \ref{eq:dphib2_} to \ref{eq:dphib1} gives
\begin{equation}
  \label{eq:dphib1_}
  \begin{split}
    \frac{\partial\phi_b}{\partial\mathbf{\sigma}}          &= 2\ \text{sgn}(\hat{\mathbf{h}}\cdot\mathbf{s}) \bigg[  \frac{\partial{f_k}}{\partial{\mathbf{\sigma}}} (\hat{\mathbf{h}}\cdot\mathbf{s}) + f_k \frac{\partial\big({\hat{\mathbf{h}}\cdot\mathbf{s}}\big)}{\partial{\mathbf{\sigma}}} \bigg]\\
    \bigg(\frac{\partial\phi_b}{\partial \sigma}\bigg)_{i} &= 2\ \text{sgn}(\hat{h}_{m}s_{m}) \bigg[  \bigg(\frac{\partial f_k}{\partial \sigma}\bigg)_{i} (\hat{h}_{m}s_{m}) + f_k \bigg(\frac{\partial\big({\hat{h}_{m}s_{m}}\big)}{\partial \sigma}\bigg)_{i} \bigg]\\
  \end{split}
\end{equation}
\subsubsection{$\partial\big(\hat{\mathbf{h}}\cdot\mathbf{s}\big)/ \partial\mathbf{\sigma}$ }
\label{sec:3.2.1}
The derivative  $\partial\big(\hat{\mathbf{h}}\cdot\mathbf{s}\big)/ \partial\mathbf{\sigma}$ is obtained as:
\begin{equation}
  \begin{split}
    \label{eq:dphib3}
    \frac{\partial \big(\hat{\mathbf{h}}\cdot\mathbf{s} \big)  } {\partial\mathbf{\sigma}} &= \frac{\partial \hat{\mathbf{h}}}{\partial\mathbf{\sigma}}\cdot\mathbf{s} +\hat{\mathbf{h}}\cdot\frac{\partial \mathbf{s}}{\partial\mathbf{\sigma}}\\
    \bigg(\frac{\partial \big(\hat{h}\cdot s\big)} {\partial\sigma}\bigg)_i &= \bigg(\frac{\partial\hat{h}}{\partial \sigma}\bigg)_{ij} s_j +\hat{h}_j\frac{\partial s}{\partial\sigma}_{ji}
  \end{split}
\end{equation}
The term $ \partial\hat{\mathbf{h}}/ \partial\mathbf{\sigma} $ can be obtained through:
\begin{equation}
  \label{eq:dphib4}
\frac{\partial\hat{\mathbf{h}}}{\partial\mathbf{\sigma}}=\frac{\partial{\hat{\mathbf{h}}}}{\partial{\bar{\varepsilon}}}   \frac{\partial{\bar{\varepsilon}}}{\partial{\mathbf{\sigma}}}
\end{equation}
$\partial\hat{\mathbf{h}}/\partial\bar{\varepsilon} $ can be obtained using the evolutionary rule of $\hat{\mathbf{h}}$.
Note that $\partial\bar{\mathbf{\varepsilon}}/\partial\mathbf{\sigma}$ can be obtained using the normality rule:
\begin{equation}
  \label{eq:dphib5}
  \bigg(\frac{\partial{\mathbf{\sigma}}}{\partial{\bar{\varepsilon}}}\bigg)_i = - \mathbb{C}^{el} \cdot \frac{\partial\Phi}{\partial{\mathbf{\sigma}}} = -\mathbb{C}^{el}_{ij} \bigg(\frac{\partial\Phi}{\partial\sigma}\bigg)_j
\end{equation}
which leads to
\begin{equation}
  \label{eq:dphib6}
  \begin{split}
      \frac{\partial\hat{\mathbf{h}}}{\partial\mathbf{\sigma}} &=\frac{\partial\hat{\mathbf{h}}}{\partial\bar{\varepsilon}}\otimes \bigg\{ \Big\{- \mathbb{C}^{el} \cdot \frac{\partial\Phi}{\partial\mathbf{\sigma}}\Big\}^{-1} \bigg\}\\
    \bigg(\frac{\partial\hat{h}}{\partial \sigma}\bigg)_{ij}&=\bigg(\frac{\partial\hat{h}}{\partial\bar{\varepsilon}}\bigg)_i \bigg\{-\mathbb{C}^{el}_{jk} \Big(\frac{\partial\Phi}{\partial\sigma}\Big)_{k}\bigg\}^{-1}
  \end{split}
\end{equation}
Substitute Eq. \ref{eq:dphib6} to Eq. \ref{eq:dphib3} to have
\begin{equation}
  \label{eq:dphib6_1}
  \begin{split}
    \frac{\partial \big(\hat{\mathbf{h}}\cdot\mathbf{s}\big)  }{\partial\mathbf{\sigma}}&=\frac{\partial{\hat{\mathbf{h}}}}{\partial{\bar{\varepsilon}}} \bigg\{ \Big\{- \mathbb{C}^{el} \cdot \frac{\partial \Phi}{\partial{\mathbf{\sigma}}}\Big\}^{-1}\cdot\mathbf{s}\bigg\} + \hat{\mathbf{h}}\cdot\frac{\partial\mathbf{s}}{\partial\mathbf{\sigma}}\\
    \bigg(\frac{\partial \big(\hat{h}\cdot s\big)}{\partial \sigma}\bigg)_i & = \bigg(\frac{\partial\hat{h}}{\partial\bar{\varepsilon}}\bigg)_i \bigg\{\bigg(\frac{\partial\bar{\varepsilon}}{\partial\sigma}\bigg)_m s_m\bigg\} + \hat{h}_m\bigg(\frac{\partial s}{\partial \sigma}\bigg)_{mi},
  \end{split}
\end{equation}
where $\partial \hat{\mathbf{h}}/ \partial\bar{\varepsilon}$ is given in Eq. \ref{eq:dmicro1}.
\subsubsection{Calculation procedure on $\big(\partial\phi_b/\partial \sigma\big)_{ij}$}
\label{sec:calc}
Substitution of Eqs. \ref{eq:dphib8} and \ref{eq:dphib6_1} into Eq. \ref{eq:dphib1_} gives below:
\begin{equation}
  \label{eq:dphib_main}
  \begin{split}
    \bigg(\frac{\partial\phi_b}{\partial \sigma}\bigg)_i &= 2\ \text{sgn}(\hat{h}_ms_m) \bigg[(\hat{h}_ms_m)
    \frac{\partial f_k}{\partial g_k} \frac{\partial g_k}{\partial \bar{\varepsilon}}
    \bigg(\frac{\partial\bar{\varepsilon}}{\partial\sigma}\bigg)_i
   \\
    &+ f_k\bigg\{
    \bigg(\frac{\partial\hat{h}}{\partial\bar{\varepsilon}}\bigg)_i \bigg\{\bigg(\frac{\partial\bar{\varepsilon}}{\partial\sigma}\bigg)_m s_m\bigg\} + \hat{h}_j \bigg(\frac{\partial s}{\partial \sigma}\bigg)_{ji}
      \bigg\}
    \bigg]
  \end{split}
\end{equation}
Note that the above equation is an implicit function of $\partial\Phi/\partial\sigma\ $(i.e., $\partial\Phi/\partial \sigma$) through $\partial\bar{\varepsilon}/\partial\sigma$ as shown in Eq. \ref{eq:dphib5}.
\newpage
\section{Calculation procedure}
\label{sec:prc}

The first derivative of the HAH yield surface is summarized as below:

\begin{equation}
  \label{eq:hah_yieldsurface_derivative}
  \begin{split}
    \dfrac{\partial\Phi(\sigma)}{\partial\sigma}&=\frac{1}{q}\Phi^{1-q}\bigg\{\phi_h^{(q-1)}\frac{\partial\phi_h}{\partial\sigma} + \phi_b^{(q-1)}\frac{\partial\phi_b}{\partial\sigma}\bigg\}\\
    &=\frac{1}{q}\Phi^{1-q}\bigg[\phi_h^{(q-1)}\frac{1}{2\phi_h}\bigg\{ 2\psi(\mathbf{s}^{\prime\prime})\frac{\partial{\psi(\mathbf{s}^{\prime\prime})}}{\partial{\mathbf{s}^{\prime\prime}}}  \frac{\partial\mathbf{s}^{\prime\prime}}{\partial \mathbf{\sigma}}+ 2\psi(\mathbf{s}_p) \frac{\partial{\psi(\mathbf{s}_p)}}{\partial{\mathbf{s}_p}} \frac{\partial{\mathbf{s}_p}}{\partial{\mathbf{\sigma}}}\bigg\} \\
    &+ \phi_b^{(q-1)}\frac{\partial\phi_b}{\partial\sigma}\bigg]
  \end{split}
\end{equation}
More explicitly, the below is obtained.
\begin{equation}
  \label{eq:hah_yieldsurface_derivative_aux}
  \begin{split}
    \dfrac{\partial\Phi(\mathbf{\sigma})}{\partial\mathbf{\sigma}}&=\frac{1}{q}\Phi^{1-q}\bigg\{\phi_h^{(q-1)}\frac{\partial\phi_h}{\partial\mathbf{\sigma}}+\phi_b^{(q-1)}\frac{\partial\phi_b}{\partial\mathbf{\sigma}}\bigg\}\\
    &=\frac{1}{q}\Phi^{1-q}\bigg[\phi_h^{(q-2)}\bigg\{\psi(\mathbf{s}^{\prime\prime})\frac{\partial{\psi(\mathbf{s}^{\prime\prime})}}{\partial{\mathbf{s}^{\prime\prime}}}\frac{\partial\mathbf{s}^{\prime\prime}}{\partial \mathbf{\sigma}}+\psi(\mathbf{s}_p)\frac{\partial{\psi(\mathbf{s}_p)}}{\partial{\mathbf{s}_p}}\frac{\partial{\mathbf{s}_p}}{\partial{\mathbf{\sigma}}}\bigg\}\\
    &+2\phi_b^{(q-1)}\text{sgn}(\hat{\mathbf{h}}\cdot\mathbf{s})\bigg\{(\hat{\mathbf{h}}\cdot\mathbf{s})
    \frac{\partial f_k}{\partial g_k} \frac{\partial g_k}{\partial\bar{\varepsilon}}
    \frac{\partial\bar{\varepsilon}}{\partial\mathbf{\sigma}}+f_k\Bigg(\frac{\partial\hat{h}}{\partial\bar{\varepsilon}}\bigg(\frac{\partial\bar{\varepsilon}}{\partial\mathbf{\sigma}}\cdot\mathbf{s}\bigg)+\hat{h}\cdot\frac{\partial\mathbf{s}}{\partial \mathbf{\sigma}}
      \Bigg)\bigg\} \bigg]
  \end{split}
\end{equation}
with $k$ being 1 or 2 depending on the sign of $\hat{\mathbf{h}}\cdot\mathbf{s}$.
Note that both $\dfrac{\partial\mathbf{s}^{\prime\prime}}{\partial\mathbf{\sigma}}$ and $\dfrac{\partial\mathbf{s}_p}{\partial\mathbf{\sigma}}$ depend on $\dfrac{\partial\bar{\varepsilon}}{\partial\mathbf{\sigma}}$.

For better understanding, Eq. \ref{eq:hah_yieldsurface_derivative_aux} is expressed as:
\begin{equation}
  \label{eq:hah_yieldsurface_derivative_short}
  \bigg(\dfrac{\partial\Phi(\mathbf{\sigma})}{\partial\mathbf{\sigma}}\bigg)_i=\frac{1}{q}\Phi^{1-q}\bigg[\phi_h^{(q-2)}\bigg\{\psi(\mathbf{s}^{\prime\prime})A_i+\psi(\mathbf{s}_p)B_i\bigg\}+2\phi_b^{(q-1)}\text{sgn}(\hat{\mathbf{h}}\cdot\mathbf{s})\bigg\{C_i+ D_i\bigg\} \bigg]
\end{equation}
where $A_{i}=\sum_j\bigg(\dfrac{\partial\phi}{\partial\mathbf{s}^{\prime\prime}}\bigg)_{j}\bigg(\dfrac{\partial\mathbf{s}^{\prime\prime}}{\partial\mathbf{\sigma}}\bigg)_{ji}$, $B_{i}=\bigg[\sum_m(\hat{h}_ms_m)\bigg]\sum_j\bigg(\dfrac{\partial\phi}{\partial\mathbf{s}_p}\bigg)_{j}\bigg(\dfrac{\partial\mathbf{s}_p}{\partial\mathbf{\sigma}}\bigg)_{ji}$,$C_i=\dfrac{\partial f_k}{\partial g_k}\dfrac{\partial g_k}{\partial \bar{\varepsilon}}\bigg(\dfrac{\partial\bar{\varepsilon}}{\partial\sigma}\bigg)_{i}$ and $D_i= f_k\bigg[\bigg(\dfrac{\partial\hat{h}}{\partial\bar{\varepsilon}}\bigg)_i\sum_m\bigg\{{\bigg(\dfrac{\partial\bar{\varepsilon}}{\partial\sigma}\bigg)_m s_m}\bigg\}+\sum_j\bigg\{\hat{h}_j\bigg(\dfrac{\partial s}{\partial\sigma}\bigg)_{ji}\bigg\} \bigg]$.
Since $\dfrac{\partial\bar{\varepsilon}}{\partial\sigma}=\bigg\{-\mathbb{C}^{el}\cdot\dfrac{\partial\Phi}{\partial\mathbf{\sigma}}\bigg\}^{-1}$, Eq. \ref{eq:hah_yieldsurface_derivative_aux} is an implicit function.
Moreover, each of $A_i, B_i, C_i$ and $D_i$ is individually dependent on $\dfrac{\partial\Phi}{\partial\sigma}$. Therefore Eq. \ref{eq:hah_yieldsurface_derivative_short} is an implicit function.
The solution of $\dfrac{\partial\Phi}{\partial\sigma}$ should be iteratively obtained.
At the very initial state, $\dfrac{\partial\Phi(\mathbf{\sigma})}{\partial\mathbf{\sigma}}=\dfrac{\partial \phi_h}{\partial\sigma}$.
To obtain the general step $n+1$, previous state variables (such as $\hat{\mathbf{h}}^{(n)}, g_L^{(n)}, g_S^{(n)}, g_1^{(n)},$ and $g_2^{(n)}$) as well as the  stress derivative of yield surface $\bigg(\dfrac{\partial\Phi(\mathbf{\sigma})}{\partial\mathbf{\sigma}}\bigg)^{(n)}$ is useful when seeking for the numerical solution of $n+1$ step.
For the currently 'guessed' plastic strain increment $\Delta\lambda^{(k)}$, the corrected equivalent plastic strain increment should be updated for the step $n+1$.
This value is subjected to the iterative estimation, during which the relevant state variables are denoted with the upper tilde symbol {\raise.17ex\hbox{$\scriptstyle\sim$}} such that
$\Delta\tilde{\lambda}^{(n+1)}=\Delta\lambda^{(n)}+\Delta\lambda^{(k)}$
Note that $\Delta\lambda^{(k=0)}$ is obtained using the elastic predictor.
Now that the $\Delta\tilde{\lambda}^{(n+1)}$ is estimated, the corresponding state variables should be obtained, which are listed as $\tilde{g_L}^{(n+1)}, \tilde{g_S}^{(n+1)}, \tilde{g_1}^{(n+1)},$ and $\tilde{g_2}^{(n+1)}$.
The calculation of $\tilde{\Phi}^{n+1}$ is rather strainghtforward as it is an explicit function of the state variables and the $\tilde{\mathbf{\sigma}}$ stress.
On the other hand, according to Eq. \ref{eq:hah_yieldsurface_derivative_aux} $\tilde{\bigg(\dfrac{\partial\Phi}{\partial\mathbf{\sigma}}\bigg)}^{(n+1)}$ is in an implicit function which requires additional iterative scheme.

This can be performed by introducing the multi-step procedure which divide the interval increments.
Say for state variable $g$, we use $g^{(n)}$ and $\tilde{g}^{(n+1)}$ to sub-divide the each k step into a separate iterative loop indexed by $l$ such that
\begin{equation}
  \label{eq:g_subdivision}
  g^{l}=g^{(n)}+\frac{\tilde{g}^{(n+1)}-n^{(n)}}{N}\cdot l \text{  where } l \text{ is 1, 2, ... , N}
\end{equation}
where $N$ denotes the total number of subdivision.
For each $l$ step to calculate $\tilde{\bigg(\dfrac{\partial\phi}{\partial\sigma}\bigg)}^{(l+1)}$ we explicitly obtain $A_i, B_i, C_i$, and $D_i$ with using $\tilde{\bigg(\dfrac{\partial\phi}{\partial\sigma}\bigg)}^{(l)}$.
Note that $\tilde{\bigg(\dfrac{\partial\phi}{\partial\sigma}\bigg)}^{(l=0)}$ is equivalent to $\bigg(\dfrac{\partial\phi}{\partial\sigma}\bigg)^{(n)}$.
For the last step of $l$ loop, we must ensure that the difference between the input $\tilde{\bigg(\dfrac{\partial\phi}{\partial\sigma}\bigg)}^{(l)}$ and the output $\tilde{\bigg(\dfrac{\partial\phi}{\partial\sigma}\bigg)}^{(l+1)}$ is marginal.
If that is the case, we update $\tilde{\bigg(\dfrac{\partial\phi}{\partial\sigma}\bigg)}^{(n+1)}=\tilde{\bigg(\dfrac{\partial\phi}{\partial\sigma}\bigg)}^{(l+1)}$.
If not, consider reducing the total number of subdivision.

We start to calculate $\cos\chi$ and $\partial \hat{\mathbf{h}}/\partial \bar{\varepsilon}$ using Eqs. \ref{eq:coschi} and \ref{eq:dmicro1}, respectively.
\begin{enumerate}
\item Use $\partial \hat{\mathbf{h}}/\partial \bar{\varepsilon}$ to obtain $\partial \hat{\mathbf{h}}/\partial \mathbf{\sigma}$ in Eq. \ref{eq:dphib6}. This also requires $\mathbb{C}^{el}$ and $\partial\Phi/\partial\mathbf{\sigma}$, which gives $\dfrac{\partial\bar{\varepsilon}}{\partial\sigma}$.
\item Obtain $\partial\big(\hat{\mathbf{h}}:\mathbf{s}\big)/\partial \mathbf{\sigma}$ in Eq. \ref{eq:dphib3}
\item Also, obtain $\partial \mathbf{s}_c/\partial \mathbf{s}$ and $\partial \mathbf{s}_o/\partial\mathbf{s}$ in Eqs. \ref{eq:decomp2} and \ref{eq:decomp4}, respectively.
\item Obtain $\partial \mathbf{s}_p/\partial\mathbf{s}$ using Eq. \ref{eq:cross_linear2}.
\item Obtain $\partial\mathbf{s}^{\prime\prime}/\partial\mathbf{s}$ using $\partial \mathbf{s}_c/\partial \mathbf{s}$.
\item Plug $\partial\mathbf{s}^{\prime\prime}/\partial\mathbf{s}$ and  $\partial \mathbf{s}_p/\partial\mathbf{s}$ into Eq. \ref{eq:derv3} to obtain $\partial\phi_h/\partial\mathbf{\sigma}$.
\item Obtain $\partial f_k/\partial \mathbf{\sigma}$ in Eq. \ref{eq:dphib8}.
\item Calculate $\partial \phi_b/\partial \mathbf{\sigma}$ in Eq. \ref{eq:dphib1_}
\item Plug $\partial \phi_b/\partial\mathbf{\sigma}$ and $\partial\phi_h/\partial\mathbf{\sigma}$ into Eq. \ref{eq:hah_deriv}.
% \item Note that both $\partial \phi_b/\partial\mathbf{\sigma}$ and $\partial\phi_h/\partial\mathbf{\sigma}$ are dependent on $\partial\Phi/\partial\mathbf{\sigma}.$
\end{enumerate}
\newpage
\section{yld2000-2d yield function}
The current section presents discuss the mathematical formalism of yld2000-2d yield surface.

By using the form of the isotropic yield function of hershey, the yield surface ($\psi$) is defined as
\begin{equation}
  \label{eq:hershey}
  \psi=\bigg(\frac{\phi^{(1)} + \phi^{(2)}}{2}\bigg)^{\frac{1}{a}}=\bar{\sigma}
\end{equation}
where
\begin{equation}
  \label{eq:hershey2}
  \phi^{(1)} = |\chi^{(1)}_1-\chi^{(1)}_2|^a \text{ and } \phi^{(2)}=|2\chi^{(2)}_2+\chi^{(2)}_1|^a+|2\chi^{(2)}_1+\chi^{(2)}_2|^a.
\end{equation}

In the above, the symbol $\chi^{(1)}_i$ and $\chi^{(2)}_i$ (with $i=1,2$) denote the two principal components of the associated tensors $\mathbf{X}^{(1)}$ and $\mathbf{X}^{(2)}$, respectively.
The tensors $\mathbf{X}^{(1)}$ and $\mathbf{X}^{(2)}$ result from the linear transformations performed on the associated stress tensor ($\mathbf{\sigma}$) using $\mathbf{L}^{(1)}$ and $\mathbf{L}^{(2)}$ tensors, respectively, which account for the anisotropy of the material.
The derivative of $\psi$ with respect to stress tensor ($\mathbf{\sigma}$) is obtained as below:
\begin{equation}
  \label{eq:derivative1}
  \begin{split}
    \frac{\partial\psi}{\partial\mathbf{\sigma}}&=\frac{1}{a}\bigg(\frac{\phi^{(1)}+\phi^{(2)}}{2}\bigg)^{\frac{1}{a}-1}\bigg(\frac{1}{2}\frac{\partial\phi^{(1)}}{\partial\mathbf{\sigma}\hfill}+\frac{1}{2}\frac{\partial\phi^{(2)}}{\partial\mathbf{\sigma}\hfill}\bigg)=\frac{1}{2a}\psi^{1-a}\bigg(\frac{\partial\phi^{(1)}}{\partial\mathbf{\sigma}\hfill}+\frac{\partial\phi^{(2)}}{\partial\mathbf{\sigma}\hfill}\bigg)\\
    \frac{\partial\psi}{\partial\sigma_i}&=\frac{1}{2a}\psi^{1-a}\bigg(\frac{\partial\phi^{(1)}}{\partial\sigma_i\hfill}+\frac{\partial\phi^{(2)}}{\partial\sigma_i\hfill}\bigg)
  \end{split}
\end{equation}

The term $\partial\phi^{(1)}/\partial\mathbf{\sigma}$ is obtained using the chain rule as follows.
\begin{equation}
  \label{eq:derivative1_1}
  \frac{\partial\phi^{(1)}}{\partial\sigma_i\hfill}=\sum_j^2\sum_k^3 \frac{\partial\phi^{(1)}}{\partial\chi^{(1)}_j}  \frac{\partial\chi^{(1)}_j}{\partial X^{(1)}_k}   \frac{\partial X^{(1)}_k}{\partial\sigma_i\hfill}
\end{equation}
The same rule applies to $\partial\phi^{(2)}/\partial\mathbf{\sigma}$.



The second derivative (Hessian matrix, i.e., $\partial^2\psi/\partial\sigma_i\partial\sigma_j$) is obtained on the basis of Eq. \ref{eq:derivative1}.
\begin{equation}
  \label{eq:derivative2}
  \begin{split}
    \frac{\partial^2\psi}{\partial\sigma_i\partial\sigma_j}=&\frac{1}{2a} (1-a)\psi^{-a}\frac{\partial\psi}{\partial\sigma_j}  \bigg\{\frac{\partial\phi^{(1)}}{\partial\sigma_i}+\frac{\partial\phi^{(2)}}{\partial\sigma_i}\bigg\}+\frac{1}{2a}\psi^{1-a}   \bigg\{\frac{\partial^2\phi^{(1)}}{\partial\sigma_i\partial\sigma_j}+\frac{\partial^2\phi^{(2)}}{\partial\sigma_i\partial\sigma_j}\bigg\}\\
    =&\frac{1}{2a} (1-a)\psi^{-a}\frac{\partial\psi}{\partial\sigma_j}  \bigg\{2a\ \psi^{a-1} \frac{\partial\psi}{\partial\sigma_i} \bigg\}                +\frac{1}{2a}\psi^{1-a}   \bigg\{\frac{\partial^2\phi^{(1)}}{\partial\sigma_i\partial\sigma_j}+\frac{\partial^2\phi^{(2)}}{\partial\sigma_i\partial\sigma_j}\bigg\}\\
    =&\frac{1-a}{\psi}\frac{\partial\psi}{\partial\sigma_i}  \frac{\partial\psi}{\partial\sigma_j}   +\frac{\psi^{1-a}}{2a} \bigg\{\frac{\partial^2\phi^{(1)}}{\partial\sigma_i\partial\sigma_j}+\frac{\partial^2\phi^{(2)}}{\partial\sigma_i\partial\sigma_j}\bigg\}
  \end{split}
\end{equation}

The term $\partial^2\phi^{(1)}/\partial\sigma_i\partial\sigma_j$ can be obtained on the basis of Eq. \ref{eq:derivative1_1}.

%%  \frac{\partial}{\partial}    \frac{\partial^2}{\partial\partial}
\begin{equation}
  \label{eq:derivative_2_1}
  \begin{split}
    \frac{\partial^2\phi^{(1)}}{\partial\sigma_i\partial\sigma_j}=&\sum_k^2 \sum_l^3 \sum_m^2 \sum_n^3 \frac{\partial^2\phi^{(1)}}{\partial\chi^{(1)}_k\partial\chi^{(1)}_m} \bigg( \frac{\partial\chi^{(1)}_k}{\partial X^{(1)}_l}    \frac{\partial X^{(1)}_l}{\partial \sigma_i}\bigg)  \bigg( \frac{\partial\chi^{(1)}_m}{\partial X^{(1)}_n}    \frac{\partial X^{(1)}_n}{\partial \sigma_j}\bigg)\\
    +&\sum_k^2 \sum_l^3 \sum_m^3 \frac{\partial\phi^{(1)} }{\partial\chi^{(1)}_k}     \frac{\partial^2\chi^{(1)}_k}{\partial X^{(1)}_l \partial  X^{(1)}_m}   \frac{\partial X^{(1)}_l}{\partial\sigma_i}    \frac{\partial X^{(1)}_m}{\partial\sigma_j}\\
    +&\sum_k^2 \sum_l^3 \frac{\partial\phi^{(1)} }{\partial\chi^{(1)}_k}   \frac{\partial\chi^{(1)}_k}{\partial X^{(1)}_l}  \frac{\partial^2 X^{(1)}_l}{\partial\sigma_i\partial\sigma_j}.
  \end{split}
\end{equation}
The same approach is needed for $\partial^2\phi^{(2)}/\partial\sigma_i\partial\sigma_j$.

a. Terms $\partial\phi^{(1)}/\partial\mathbf{\chi}^{(1)}$ and $\partial\phi^{(2)}/\partial\mathbf{\chi}^{(2)}$ are derived from Eq. \ref{eq:hershey2} as:
\begin{equation}
  \label{eq:derivative_2_2}
  \begin{split}
    \begin{bmatrix}
      \partial\phi^{(1)}/\partial\mathbf{\chi}^{(1)}_1\\
      \partial\phi^{(1)}/\partial\mathbf{\chi}^{(1)}_2
    \end{bmatrix}
    =&
    \begin{bmatrix}
      a(\chi^{(1)}_1-\chi^{(1)}_2)^{a-1}\\
      -a(\chi^{(1)}_1-\chi^{(1)}_2)^{a-1}
    \end{bmatrix}\\
    \begin{bmatrix}
      \partial\phi^{(2)}/\partial\mathbf{\chi}^{(2)}_1\\
      \partial\phi^{(2)}/\partial\mathbf{\chi}^{(2)}_2
    \end{bmatrix}
    =&
    \begin{bmatrix}
      a|2\chi_2^{(2)}+\chi_1^{(2)}|^{a-1} \text{ sign}(2\chi_2^{(2)}+\chi_1^{(2)})+2a|2\chi_1^{(2)}+\chi_2^{(2)}|^{a-1} \text{ sign}(2\chi_1^{(2)}+\chi_2^{(2)})\\
      2a|2\chi_2^{(2)}+\chi_1^{(2)}|^{a-1} \text{ sign}(2\chi_2^{(2)}+\chi_1^{(2)})+ a|2\chi_1^{(2)}+\chi_2^{(2)}|^{a-1} \text{ sign}(2\chi_1^{(2)}+\chi_2^{(2)})
    \end{bmatrix}
  \end{split}
\end{equation}
With the above, the second order derivatives ($\partial^2\phi^{(1)}/\partial\chi^{(1)}_i\partial\chi^{(1)}_j$ and $\partial^2\phi^{(2)}/\partial\chi^{(2)}_i\partial\chi^{(2)}_i$) are obtained as below:
\begin{equation}
  \label{eq:derivative_2_3}
  \begin{split}
    \begin{bmatrix}
      \partial^2\phi^{(1)}/\partial\mathbf{\chi}^{(1)}_1\partial\mathbf{\chi}^{(1)}_1\\
      \partial^2\phi^{(1)}/\partial\mathbf{\chi}^{(1)}_1\partial\mathbf{\chi}^{(1)}_2\\
      \partial^2\phi^{(1)}/\partial\mathbf{\chi}^{(1)}_2\partial\mathbf{\chi}^{(1)}_1\\
      \partial^2\phi^{(1)}/\partial\mathbf{\chi}^{(1)}_2\partial\mathbf{\chi}^{(1)}_2
    \end{bmatrix}
    &=
    \begin{bmatrix}
      a(a-1)(\chi^{(1)}_1-\chi^{(1)}_2)^{a-2}\\
      -a(a-1)(\chi^{(1)}_1-\chi^{(1)}_2)^{a-2}\\
      -a(a-1)(\chi^{(1)}_1-\chi^{(1)}_2)^{a-2}\\
      a(a-1)(\chi^{(1)}_1-\chi^{(1)}_2)^{a-2}
    \end{bmatrix}\\
  \end{split}
\end{equation}
\begin{equation}
  \label{eq:derivative_2_4}
  \begin{split}
    \begin{bmatrix}
      \partial^2\phi^{(2)}/\partial\mathbf{\chi}^{(2)}_1\partial\mathbf{\chi}^{(2)}_1\\
      \partial^2\phi^{(2)}/\partial\mathbf{\chi}^{(2)}_1\partial\mathbf{\chi}^{(2)}_2\\
      \partial^2\phi^{(2)}/\partial\mathbf{\chi}^{(2)}_2\partial\mathbf{\chi}^{(2)}_1\\
      \partial^2\phi^{(2)}/\partial\mathbf{\chi}^{(2)}_2\partial\mathbf{\chi}^{(2)}_2
    \end{bmatrix}
    &=a(a-1)
    \begin{bmatrix}
      |2\chi^{(2)}_2+\chi^{(2)}_1|^{a-2}+4|2\chi^{(2)}_1+\chi^{(2)}_2|^{a-2}\\
      2|2\chi^{(2)}_2+\chi^{(2)}_1|^{a-2}+2|2\chi^{(2)}_1+\chi^{(2)}_2|^{a-2}\\
      2|2\chi^{(2)}_2+\chi^{(2)}_1|^{a-2}+2|2\chi^{(2)}_1+\chi^{(2)}_2|^{a-2}\\
      4|2\chi^{(2)}_2+\chi^{(2)}_1|^{a-2}+|2\chi^{(2)}_1+\chi^{(2)}_2|^{a-2}
    \end{bmatrix}
  \end{split}
\end{equation}
\newline
b. Terms $\partial\chi^{(1)}_i/\partial X^{(1)}_j$ is obtained on the basis of the below equation.
\begin{equation}
  \label{eq:princ}
  \begin{split}
    \chi_1^{(k)}=\frac{1}{2}\bigg(X_1^{(k)}+X_2^{(k)}+\sqrt{(X_1^{(k)}-X_2^{(k)})^2+4(X_3^{(k)})^2}\bigg)\\
    \chi_2^{(k)}=\frac{1}{2}\bigg(X_1^{(k)}+X_2^{(k)}-\sqrt{(X_1^{(k)}-X_2^{(k)})^2+4(X_3^{(k)})^2}\bigg)\\
    \text{ with k being 1 or 2}
  \end{split}
\end{equation}
The derivatives ($\partial\chi_i^{(k)}/\partial X_j^{(k)}$) are found in what follows.
We first consider $k=1$.
With $\Delta=\big(X^{(1)}_1-X^{(1)}_2\big)^2 +4\big(X^{(1)}_3\big)^2$,
\begin{equation}
  \label{eq:princ1}
  \begin{split}
    \chi_1^{(k)}=\frac{1}{2}\bigg(X_1^{(k)}+X_2^{(k)}+\sqrt{\Delta}\bigg), \chi_2^{(k)}=\frac{1}{2}\bigg(X_1^{(k)}+X_2^{(k)}-\sqrt{\Delta}\bigg)\\
    \text{ with k being 1 or 2}
  \end{split}
\end{equation}
There are a few useful results as below:
\begin{equation}
  \label{eq:princ2}
  \begin{split}
    \frac{\partial\Delta}{\partial X_1}=2(X_1-X_2),\ \frac{\partial\Delta}{\partial X_2}=-2(X_1-X_2)=-\frac{\partial\Delta}{\partial X_1},\text{ and } \frac{\partial\Delta}{\partial X_3}=8X_3\\
  \end{split}
\end{equation}
A few more useful results for second order derivatives are listed below.
\begin{equation}
  \label{eq:princ3}
  \begin{bmatrix}
    \dfrac{\partial^2\Delta}{\partial X_1 \partial X_1} & \dfrac{\partial^2\Delta}{\partial X_1 \partial X_2} & \dfrac{\partial^2\Delta}{\partial X_1 \partial X_3} \\
    \dfrac{\partial^2\Delta}{\partial X_2 \partial X_1} & \dfrac{\partial^2\Delta}{\partial X_2 \partial X_2} & \dfrac{\partial^2\Delta}{\partial X_2 \partial X_3} \\
    \dfrac{\partial^2\Delta}{\partial X_3 \partial X_1} & \dfrac{\partial^2\Delta}{\partial X_3 \partial X_2} & \dfrac{\partial^2\Delta}{\partial X_3 \partial X_3}
  \end{bmatrix}
  =
  \begin{bmatrix}
    2& -2& 0\\
   -2&  2& 0\\
    0&  0& 8
  \end{bmatrix}
\end{equation}
The first derivative $\partial\chi^{(1)}_i/\partial X^{(1)}_j$  is obtained as below:
\begin{equation}
  \label{eq:dchi_dx_1}
  \begin{bmatrix}
    \partial\chi^{(1)}_1/\partial X^{(1)}_1\\
    \partial\chi^{(1)}_1/\partial X^{(1)}_2\\
    \partial\chi^{(1)}_1/\partial X^{(1)}_3\\
    \partial\chi^{(1)}_2/\partial X^{(1)}_1\\
    \partial\chi^{(1)}_2/\partial X^{(1)}_2\\
    \partial\chi^{(1)}_2/\partial X^{(1)}_3
  \end{bmatrix}
  =
  \frac{1}{2}
  \begin{bmatrix}
     1 +  \frac{1}{2}\frac{\partial\Delta}{\partial X_1} \Delta^{(-1/2)}   \\
     1 +  \frac{1}{2}\frac{\partial\Delta}{\partial X_2} \Delta^{(-1/2)}   \\
          \frac{1}{2}\frac{\partial\Delta}{\partial X_3} \Delta^{(-1/2)}   \\
     1 -  \frac{1}{2}\frac{\partial\Delta}{\partial X_1} \Delta^{(-1/2)}   \\
     1 -  \frac{1}{2}\frac{\partial\Delta}{\partial X_2} \Delta^{(-1/2)}   \\
         -\frac{1}{2}\frac{\partial\Delta}{\partial X_3} \Delta^{(-1/2)}
  \end{bmatrix}
  =
  \frac{1}{2}
  \begin{bmatrix}
     1 +  \frac{1}{2}\frac{\partial\Delta}{\partial X_1} \Delta^{(-1/2)}   \\
     1 -  \frac{1}{2}\frac{\partial\Delta}{\partial X_1} \Delta^{(-1/2)}   \\
          \frac{1}{2}\frac{\partial\Delta}{\partial X_3} \Delta^{(-1/2)}   \\
     1 -  \frac{1}{2}\frac{\partial\Delta}{\partial X_1} \Delta^{(-1/2)}   \\
     1 +  \frac{1}{2}\frac{\partial\Delta}{\partial X_1} \Delta^{(-1/2)}   \\
       -  \frac{1}{2}\frac{\partial\Delta}{\partial X_3} \Delta^{(-1/2)}
  \end{bmatrix}
  =
  \begin{bmatrix}
     &\partial\chi^{(1)}_1/\partial X^{(1)}_1\\
     &\partial\chi^{(1)}_1/\partial X^{(1)}_2\\
     &\partial\chi^{(1)}_1/\partial X^{(1)}_3\\
     &\partial\chi^{(1)}_1/\partial X^{(1)}_2\\
     &\partial\chi^{(1)}_1/\partial X^{(1)}_1\\
    -&\partial\chi^{(1)}_1/\partial X^{(1)}_3
  \end{bmatrix}
\end{equation}
where the third equality can be useful later.
The same applies to $ \partial\chi^{(2)}_i/\partial X^{(2)}_j$.
Now, the second derivative ($\partial^2\chi_1^{(1)}/\partial X^{(1)}_i\partial X^{(1)}_j$) is as below:
\begin{equation}
  \label{eq:d2chi_dxdx}
  \begin{split}
    &\begin{bmatrix}
      \frac{\partial^2\chi^{(1)}_1}{\partial X^{(1)}_1\partial X^{(1)}_1}\\
      \frac{\partial^2\chi^{(1)}_1}{\partial X^{(1)}_1\partial X^{(1)}_2}\\
      \frac{\partial^2\chi^{(1)}_1}{\partial X^{(1)}_1\partial X^{(1)}_3}\\
      \frac{\partial^2\chi^{(1)}_1}{\partial X^{(1)}_2\partial X^{(1)}_1}\\
      \frac{\partial^2\chi^{(1)}_1}{\partial X^{(1)}_2\partial X^{(1)}_2}\\
      \frac{\partial^2\chi^{(1)}_1}{\partial X^{(1)}_2\partial X^{(1)}_3}\\
      \frac{\partial^2\chi^{(1)}_1}{\partial X^{(1)}_3\partial X^{(1)}_1}\\
      \frac{\partial^2\chi^{(1)}_1}{\partial X^{(1)}_3\partial X^{(1)}_2}\\
      \frac{\partial^2\chi^{(1)}_1}{\partial X^{(1)}_3\partial X^{(1)}_3}
    \end{bmatrix}
    =\dfrac{1}{4}
    \begin{bmatrix}
      \frac{\partial^2\Delta}{\partial X_1^{(1)} \partial X_1^{(1)}} \Delta^{-\frac{1}{2}} - \frac{1}{2}\frac{\partial\Delta}{\partial X_1^{(1)}} \frac{\partial\Delta}{\partial X_1^{(1)}} \Delta^{-\frac{3}{2}}\\
      \frac{\partial^2\Delta}{\partial X_1^{(1)} \partial X_2^{(1)}} \Delta^{-\frac{1}{2}} - \frac{1}{2}\frac{\partial\Delta}{\partial X_1^{(1)}} \frac{\partial\Delta}{\partial X_2^{(1)}} \Delta^{-\frac{3}{2}}\\
      \frac{\partial^2\Delta}{\partial X_1^{(1)} \partial X_3^{(1)}} \Delta^{-\frac{1}{2}} - \frac{1}{2}\frac{\partial\Delta}{\partial X_1^{(1)}} \frac{\partial\Delta}{\partial X_3^{(1)}} \Delta^{-\frac{3}{2}}\\
      \frac{\partial^2\Delta}{\partial X_2^{(1)} \partial X_1^{(1)}} \Delta^{-\frac{1}{2}} - \frac{1}{2}\frac{\partial\Delta}{\partial X_2^{(1)}} \frac{\partial\Delta}{\partial X_1^{(1)}} \Delta^{-\frac{3}{2}}\\
      \frac{\partial^2\Delta}{\partial X_2^{(1)} \partial X_2^{(1)}} \Delta^{-\frac{1}{2}} - \frac{1}{2}\frac{\partial\Delta}{\partial X_2^{(1)}} \frac{\partial\Delta}{\partial X_2^{(1)}} \Delta^{-\frac{3}{2}}\\
      \frac{\partial^2\Delta}{\partial X_2^{(1)} \partial X_3^{(1)}} \Delta^{-\frac{1}{2}} - \frac{1}{2}\frac{\partial\Delta}{\partial X_2^{(1)}} \frac{\partial\Delta}{\partial X_3^{(1)}} \Delta^{-\frac{3}{2}}\\
      \frac{\partial^2\Delta}{\partial X_3^{(1)} \partial X_1^{(1)}} \Delta^{-\frac{1}{2}} - \frac{1}{2}\frac{\partial\Delta}{\partial X_3^{(1)}} \frac{\partial\Delta}{\partial X_1^{(1)}} \Delta^{-\frac{3}{2}}\\
      \frac{\partial^2\Delta}{\partial X_3^{(1)} \partial X_2^{(1)}} \Delta^{-\frac{1}{2}} - \frac{1}{2}\frac{\partial\Delta}{\partial X_3^{(1)}} \frac{\partial\Delta}{\partial X_2^{(1)}} \Delta^{-\frac{3}{2}}\\
      \frac{\partial^2\Delta}{\partial X_3^{(1)} \partial X_3^{(1)}} \Delta^{-\frac{1}{2}} - \frac{1}{2}\frac{\partial\Delta}{\partial X_3^{(1)}} \frac{\partial\Delta}{\partial X_3^{(1)}} \Delta^{-\frac{3}{2}}
    \end{bmatrix}\\
  \end{split}
\end{equation}

Using Eq. \ref{eq:dchi_dx_1}, $\dfrac{\partial^2\chi_2^{(1)}}{\partial X_i^{(1)} \partial X_j^{(1)}}$ can be related to components of  $\dfrac{\partial^2\chi_1^{(1)}}{\partial X_i^{(1)} \partial X_j^{(1)}}$ as shown in Eq. \ref{eq:d2chi_dxdx}.

\begin{equation}
  \label{eq:d2chi_dxdx_2}
  \begin{split}
    \begin{bmatrix}
      \dfrac{\partial^2\chi^{(1)}_2}{\partial X^{(1)}_1\partial X^{(1)}_1}&\dfrac{\partial^2\chi^{(1)}_2}{\partial X^{(1)}_1\partial X^{(1)}_2}&\dfrac{\partial^2\chi^{(1)}_2}{\partial X^{(1)}_1\partial X^{(1)}_3}\\
      \dfrac{\partial^2\chi^{(1)}_2}{\partial X^{(1)}_2\partial X^{(1)}_1}&\dfrac{\partial^2\chi^{(1)}_2}{\partial X^{(1)}_2\partial X^{(1)}_2}&\dfrac{\partial^2\chi^{(1)}_2}{\partial X^{(1)}_2\partial X^{(1)}_3}\\
      \dfrac{\partial^2\chi^{(1)}_2}{\partial X^{(1)}_3\partial X^{(1)}_1}&\dfrac{\partial^2\chi^{(1)}_2}{\partial X^{(1)}_3\partial X^{(1)}_2}&\dfrac{\partial^2\chi^{(1)}_2}{\partial X^{(1)}_3\partial X^{(1)}_3}
    \end{bmatrix}
    =\\
    \begin{bmatrix}
      \dfrac{\partial^2\chi^{(1)}_1}{\partial X^{(1)}_2\partial X^{(1)}_1}&\dfrac{\partial^2\chi^{(1)}_1}{\partial X^{(1)}_2\partial X^{(1)}_2}&\dfrac{\partial^2\chi^{(1)}_1}{\partial X^{(1)}_2\partial X^{(1)}_3}\\
      \dfrac{\partial^2\chi^{(1)}_1}{\partial X^{(1)}_1\partial X^{(1)}_1}&\dfrac{\partial^2\chi^{(1)}_1}{\partial X^{(1)}_1\partial X^{(1)}_2}&\dfrac{\partial^2\chi^{(1)}_1}{\partial X^{(1)}_1\partial X^{(1)}_3}\\
      -\dfrac{\partial^2\chi^{(1)}_1}{\partial X^{(1)}_3\partial X^{(1)}_1}&-\dfrac{\partial^2\chi^{(1)}_1}{\partial X^{(1)}_3\partial X^{(1)}_2}&-\dfrac{\partial^2\chi^{(1)}_1}{\partial X^{(1)}_3\partial X^{(1)}_3}
    \end{bmatrix}
  \end{split}
\end{equation}
The formalism in Eqs. \ref{eq:d2chi_dxdx} and \ref{eq:d2chi_dxdx_2} applies to $\dfrac{\partial^2\chi_i^{(2)}}{\partial X_j^{(2)} \partial X_k^{(2)}}$.
c. $\dfrac{\partial^2 X^{(1)}_i}{\partial\sigma_j\partial\sigma_k}$
Note that $\dfrac{\partial X^{(1)}_i}{\partial\sigma_j}$ are anisotropic parameters that are constant so that $\dfrac{\partial^2 X^{(1)}_i}{\partial\sigma_j\partial\sigma_k}$ lead to zeros.
The same applies to $\dfrac{\partial^2 X^{(2)}_i}{\partial\sigma_j\partial\sigma_k}$.
Therefore, Eq. \ref{eq:derivative_2_1} reduces to
\begin{equation}
  \label{eq:derivative_2_5}
  \begin{split}
    \frac{\partial^2\phi^{(1)}}{\partial\sigma_i\partial\sigma_j}=&\sum_k^2 \sum_l^3 \sum_m^2 \sum_n^3 \frac{\partial^2\phi^{(1)}}{\partial\chi^{(1)}_k\partial\chi^{(1)}_m} \bigg( \frac{\partial\chi^{(1)}_k}{\partial X^{(1)}_l}    \frac{\partial X^{(1)}_l}{\partial \sigma_i}\bigg)  \bigg( \frac{\partial\chi^{(1)}_m}{\partial X^{(1)}_n}    \frac{\partial X^{(1)}_n}{\partial \sigma_j}\bigg)\\
    +&\sum_k^2 \sum_l^3 \sum_m^3 \frac{\partial\phi^{(1)} }{\partial\chi^{(1)}_k}     \frac{\partial^2\chi^{(1)}_k}{\partial X^{(1)}_l \partial  X^{(1)}_m}   \frac{\partial X^{(1)}_l}{\partial\sigma_i}    \frac{\partial X^{(1)}_m}{\partial\sigma_j}.
  \end{split}
\end{equation}

\newpage
\bibliography{bib}
\bibliographystyle{ieeetr}

\end{document}
